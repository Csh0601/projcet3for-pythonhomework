\documentclass[10pt,a4paper]{article}
\usepackage[UTF8]{ctex}

% ==================== 页面设置 ====================
\usepackage[margin=1.8cm, top=2cm, bottom=1.8cm]{geometry}
\usepackage{setspace}
\setstretch{1.0}
\setlength{\parskip}{0.15em}
\setlength{\headheight}{14.5pt}

% ==================== 代码高亮 ====================
\usepackage{listings}
\usepackage{xcolor}

\definecolor{codegreen}{rgb}{0,0.6,0}
\definecolor{codegray}{rgb}{0.5,0.5,0.5}
\definecolor{codepurple}{rgb}{0.58,0,0.82}
\definecolor{backcolour}{rgb}{0.95,0.95,0.95}
\definecolor{myred}{rgb}{0.8,0,0}
\definecolor{myblue}{rgb}{0.1,0.3,0.6}

\lstdefinestyle{mystyle}{
    backgroundcolor=\color{backcolour},
    commentstyle=\color{codegreen},
    keywordstyle=\color{blue},
    numberstyle=\tiny\color{codegray},
    stringstyle=\color{codepurple},
    basicstyle=\ttfamily\tiny,
    breaklines=true,
    keepspaces=true,
    numbers=left,
    numbersep=2pt,
    tabsize=2,
    language=Python,
    frame=single,
    xleftmargin=0.8em,
    framexleftmargin=0.4em
}
\lstset{style=mystyle}

% ==================== 图表支持 ====================
\usepackage{graphicx}
\usepackage{tikz}
\usetikzlibrary{shapes.geometric, arrows, positioning, fit, calc, shapes.arrows}

% ==================== 超链接 ====================
\usepackage{hyperref}
\hypersetup{colorlinks=true, linkcolor=blue, urlcolor=cyan}

% ==================== 其他 ====================
\usepackage{enumitem}
\usepackage{booktabs}
\usepackage{fancyhdr}
\usepackage{float}
\usepackage{multicol}
\usepackage{tabularx}
\usepackage{subcaption}
\usepackage{caption}

% 紧凑列表
\setlist{nosep, leftmargin=1.5em}

% 紧凑图表标题
\captionsetup{font=footnotesize, skip=2pt}

% 减少浮动体间距
\setlength{\intextsep}{2pt plus 1pt minus 1pt}
\setlength{\floatsep}{2pt plus 1pt minus 1pt}
\setlength{\textfloatsep}{4pt plus 1pt minus 2pt}

% 页眉页脚
\pagestyle{fancy}
\fancyhf{}
\rhead{\small Project 3: Iris Classification}
\lhead{\small Python Programming}
\rfoot{\small Page \thepage}

% ============================================
\begin{document}

% ==================== 紧凑标题头 ====================
\begin{center}
{\LARGE\bfseries Python程序设计 \quad 实验报告}\\[0.25cm]
{\large\bfseries Project 3: 鸢尾花数据集分类与可视化}\\[0.25cm]
\begin{tabular}{cccc}
\textbf{姓名:}陈铄涵 & \textbf{学号:}2024140014 & \textbf{课程:}Python程序设计 & \textbf{日期:}2025/12/12
\end{tabular}
\end{center}
\vspace{-0.2cm}
\hrule
\vspace{0.3cm}

% ==================== 正文开始 ====================
\section{项目概述与工作内容}

本项目基于经典的鸢尾花(Iris)数据集,\textbf{\textcolor{myred}{独立设计并实现了完整的分类与可视化系统}},涵盖四种分类任务:\textbf{四特征三分类}(数据探索)、\textbf{两特征三分类}(2D决策边界)、\textbf{三特征两分类}(3D二分类边界)、\textbf{三特征三分类}(3D三分类边界)。

\vspace{0.1cm}
\begin{figure}[H]
\centering
\begin{tikzpicture}[
    node distance=0.5cm,
    startstop/.style={rectangle, rounded corners, minimum width=1.5cm, minimum height=0.5cm, text centered, draw=black, fill=red!20, font=\scriptsize},
    process/.style={rectangle, minimum width=1.5cm, minimum height=0.5cm, text centered, draw=black, fill=blue!15, font=\scriptsize},
    myprocess/.style={rectangle, minimum width=1.5cm, minimum height=0.5cm, text centered, draw=red!70, fill=red!10, line width=1pt, font=\scriptsize\bfseries},
    arrow/.style={thick,->,>=stealth}
]
    \node (start) [startstop] {数据加载};
    \node (feat) [process, right=of start] {特征分析};
    \node (train) [process, right=of feat] {模型训练};
    \node (vis2d) [myprocess, right=of train] {2D可视化};
    \node (vis3d) [myprocess, right=of vis2d] {3D可视化};
    \node (report) [startstop, right=of vis3d] {报告生成};
    
    \draw [arrow] (start) -- (feat);
    \draw [arrow] (feat) -- (train);
    \draw [arrow] (train) -- (vis2d);
    \draw [arrow] (vis2d) -- (vis3d);
    \draw [arrow] (vis3d) -- (report);
\end{tikzpicture}
\caption{项目工作流程图(红框为本人重点实现:决策边界可视化模块)}
\end{figure}

\vspace{-0.2cm}
% ==================== 项目框架 ====================
\section{项目框架与系统架构}

\vspace{-0.1cm}
\begin{figure}[H]
\centering
\begin{tikzpicture}[
    node distance=0.35cm and 0.5cm,
    config/.style={rectangle, rounded corners, minimum width=2.5cm, minimum height=0.7cm, text centered, align=center, draw=black, fill=orange!30, font=\tiny\ttfamily},
    module/.style={rectangle, rounded corners, minimum width=2cm, minimum height=0.6cm, text centered, align=center, draw=black, fill=blue!20, font=\tiny\ttfamily},
    mymodule/.style={rectangle, rounded corners, minimum width=2cm, minimum height=0.6cm, text centered, align=center, draw=red!70, fill=red!15, line width=1pt, font=\tiny\ttfamily},
    file/.style={rectangle, minimum width=1.8cm, minimum height=0.35cm, text centered, draw=black, fill=gray!10, font=\tiny\ttfamily},
    myfile/.style={rectangle, minimum width=1.8cm, minimum height=0.35cm, text centered, draw=red!70, fill=red!10, line width=0.8pt, font=\tiny\ttfamily},
    main/.style={rectangle, rounded corners, minimum width=2.2cm, minimum height=0.45cm, text centered, draw=black, fill=green!20, font=\tiny\ttfamily},
    arrow/.style={->, thick, >=stealth}
]
    % 顶层配置
    \node[config] (config) {config.py\\(全局配置)};
    
    % 三个模块
    \node[module] (analysis) [below left=0.5cm and 1.3cm of config] {src/analysis/\\数据分析};
    \node[mymodule] (visual) [below=0.5cm of config] {src/visualization/\\可视化模块};
    \node[module] (utils) [below right=0.5cm and 1.3cm of config] {src/utils/\\工具模块};
    
    % analysis下的文件
    \node[myfile] (f1) [below=0.25cm of analysis] {1\_feature\_distribution};
    \node[myfile] (f2) [below=0.12cm of f1] {2\_feature\_analysis};
    
    % visualization下的文件
    \node[myfile] (f3) [below=0.25cm of visual] {3\_2d\_decision\_boundary};
    \node[myfile] (f4) [below=0.12cm of f3] {4\_2d\_probability\_map};
    \node[myfile] (f5) [below=0.12cm of f4] {5\_3d\_boundary\_binary};
    \node[myfile] (f6) [below=0.12cm of f5] {6\_3d\_probability\_binary};
    \node[myfile] (f7) [below=0.12cm of f6] {7\_3d\_boundary\_multiclass};
    
    % utils下的文件
    \node[file] (u1) [below=0.25cm of utils] {dataloading.py};
    \node[file] (u2) [below=0.12cm of u1] {model\_build\_train};
    \node[file] (u3) [below=0.12cm of u2] {evaluation\_metrics};
    
    % 主程序
    \node[main] (runner) [below=2.4cm of config] {8\_main\_runner.py};
    \node[main] (report) [below=0.2cm of runner] {report\_generator.py};
    
    % 连接线
    \draw [arrow] (config) -- (analysis);
    \draw [arrow] (config) -- (visual);
    \draw [arrow] (config) -- (utils);
    \draw [arrow] (analysis) -- (f1);
    \draw [arrow] (visual) -- (f3);
    \draw [arrow] (utils) -- (u1);
    % 从f7右侧绕行到runner右侧,不贯穿模块
    \draw [arrow, dashed] (f7.east) -- ++(1.2,0) |- (runner.east);
    \draw [arrow] (runner) -- (report);
    
    % 图例
    \node[font=\tiny] at (5,-0.2) {\textcolor{red}{红框}=重点实现};
    \node[font=\tiny] at (5,-0.5) {\textcolor{orange!70!black}{橙色}=核心配置};
    \node[font=\tiny] at (5,-0.8) {\textcolor{green!50!black}{绿色}=主程序};
\end{tikzpicture}
\caption{项目架构图(共11个核心文件,\textcolor{myred}{本人重点实现8个})}
\end{figure}

% ==================== 核心代码 ====================
\vspace{-0.3cm}
\section{核心代码与技术亮点}

\vspace{-0.1cm}
\begin{multicols}{2}
\subsection*{\textcolor{myred}{1. 科研级深色样式}}
\begin{lstlisting}
def set_dark_style():
  plt.style.use('dark_background')
  plt.rcParams.update({
    'figure.facecolor': '#0d0d1a',
    'axes.facecolor': '#0d0d1a',
  })
\end{lstlisting}

\subsection*{\textcolor{myred}{2. 决策边界网格}}
\begin{lstlisting}
def create_mesh(X, resolution=200):
  xx, yy = np.meshgrid(
    np.linspace(x_min, x_max, resolution),
    np.linspace(y_min, y_max, resolution))
  return xx, yy
\end{lstlisting}
\end{multicols}

\vspace{-0.4cm}
\begin{multicols}{2}
\subsection*{\textcolor{myred}{3. 3D等值面 (Marching Cubes)}}
\begin{lstlisting}
from skimage import measure
verts, faces, _, _ = measure.marching_cubes(
    prob_volume, level=0.5)
mesh = Poly3DCollection(verts[faces])
ax.add_collection3d(mesh)
\end{lstlisting}

\subsection*{\textcolor{myred}{4. 分类器配置}}
\begin{lstlisting}
CLASSIFIERS = {
  'Logistic Regression': LogisticRegression(),
  'SVM (RBF)': SVC(kernel='rbf', probability=True),
  'Random Forest': RandomForestClassifier(),
  'Gradient Boosting': HistGradientBoostingClassifier(),
}
\end{lstlisting}
\end{multicols}

% ==================== 可视化结果 ====================
\vspace{-0.3cm}
\section{可视化结果展示}

本项目实现了四种分类任务的可视化,分别展示如下:

\vspace{-0.1cm}
\subsection{任务1:四特征三分类(数据探索)}

使用全部4个特征(Sepal Length/Width, Petal Length/Width)对3个鸢尾花品种进行探索性分析。图\ref{fig:task1}左图展示四特征分布(小提琴图+箱线图+散点图),Petal特征呈双峰分布,表明Setosa与其他类差异显著;右图为Pearson相关系数矩阵,Petal Length与Width强正相关($r=0.96$)。

\begin{figure}[H]
\centering
\begin{subfigure}[b]{0.54\textwidth}
    \includegraphics[width=\textwidth]{figures/feature_distribution_combined.png}
\end{subfigure}
\hfill
\begin{subfigure}[b]{0.44\textwidth}
    \includegraphics[width=\textwidth]{figures/correlation_heatmap.png}
\end{subfigure}
\caption{四特征三分类数据探索:左-特征分布分析;右-相关性热力图}
\label{fig:task1}
\end{figure}

\vspace{-0.3cm}
\subsection{任务2:两特征三分类(2D决策边界)}

选取最具区分力的2个特征(Petal Length, Petal Width)进行三分类。图\ref{fig:task2}展示四种分类器的2D决策边界与概率分布对比:Logistic Regression产生线性边界;SVM-RBF和Random Forest捕捉非线性模式;所有分类器对Setosa完美分类,Versicolor与Virginica交界处存在不确定性区域。

\begin{figure}[H]
\centering
\includegraphics[width=0.78\textwidth]{figures/decision_boundary_2d_comparison.png}
\caption{两特征三分类:四种分类器2D决策边界与概率图对比(Petal Length vs Petal Width)}
\label{fig:task2}
\end{figure}

\vspace{-0.3cm}
\subsection{任务3:三特征两分类(3D二分类边界)}

使用3个特征(Sepal Width, Petal Length, Petal Width)进行两分类任务。图\ref{fig:task3}展示两个二分类场景:左图为Setosa vs Others(边界清晰,线性可分);右图为Versicolor vs Virginica(边界复杂,需非线性分类器)。采用Marching Cubes算法提取3D等值面。

\begin{figure}[H]
\centering
\begin{subfigure}[b]{0.48\textwidth}
    \includegraphics[width=\textwidth]{figures/3d_boundary_setosa_vs_others.png}
    \caption{Setosa vs Others}
\end{subfigure}
\hfill
\begin{subfigure}[b]{0.48\textwidth}
    \includegraphics[width=\textwidth]{figures/3d_boundary_versicolor_vs_virginica.png}
    \caption{Versicolor vs Virginica}
\end{subfigure}
\caption{三特征两分类:3D决策边界可视化(Marching Cubes等值面提取)}
\label{fig:task3}
\end{figure}

\vspace{-0.3cm}
\subsection{任务4:三特征三分类(3D三分类边界)}

使用3个特征同时对3个类别进行分类。图\ref{fig:task4}展示3D三分类综合视图:左上为3D散点图,其他三图为不同平面切片,清晰展示三类的空间分布与决策边界交汇区域。

\begin{figure}[H]
\centering
\begin{minipage}[b]{0.58\textwidth}
    \centering
    \includegraphics[width=\textwidth]{figures/3d_multiclass_combined.png}
    \captionof{figure}{三特征三分类:3D综合视图(散点图+切片)}
    \label{fig:task4}
\end{minipage}
\hfill
\begin{minipage}[b]{0.40\textwidth}
    \centering\footnotesize
    \captionof{table}{分类器性能(5折交叉验证)}
    \vspace{0.1cm}
    \begin{tabular}{lcc}
    \toprule
    \textbf{分类器} & \textbf{Acc} & \textbf{F1} \\
    \midrule
    Logistic Reg. & 97.3\% & 97.3\% \\
    SVM (RBF) & 96.7\% & 96.7\% \\
    Random Forest & 96.7\% & 96.7\% \\
    Grad. Boost & 94.7\% & 94.6\% \\
    \bottomrule
    \end{tabular}
    \label{tab:performance}
    \vspace{0.2cm}
    
    \small 所有分类器均达到94\%+准确率,验证了可视化方法的有效性。
\end{minipage}
\end{figure}

% ==================== 结论 ====================
\vspace{-0.4cm}
\section{结论与总结}

\begin{enumerate}
    \item \textbf{特征重要性}:Petal Length/Width是区分品种的关键特征,仅用两特征即可达95\%+准确率。
    \item \textbf{类别可分性}:Setosa线性可分;Versicolor与Virginica边界模糊,是分类难点。
    \item \textbf{可视化价值}:2D/3D决策边界可视化直观展示分类器决策逻辑,识别不确定性区域。
    \item \textbf{技术创新}:采用Marching Cubes算法提取3D等值面,实现科研级深色可视化风格。
\end{enumerate}

\noindent\textbf{技术特点:}模块化架构 | 四种分类任务 | Marching Cubes 3D等值面 | 科研级可视化风格

% ==================== 运行说明 ====================
\vspace{-0.2cm}
\section{运行说明}

\begin{lstlisting}[language=bash]
pip install -r requirements.txt     # 安装依赖
python src/8_main_runner.py         # 一键生成所有可视化
\end{lstlisting}

% ==================== 教师评语 ====================
\vspace{0.2cm}
\section{教师评语}

\noindent\fbox{\parbox{\dimexpr\textwidth-2\fboxsep-2\fboxrule\relax}{
\vspace{2cm}
\hfill \textbf{评分:}\underline{\hspace{2cm}} \quad \textbf{日期:}\underline{\hspace{3cm}}
\vspace{0.2cm}
}}

\end{document}
