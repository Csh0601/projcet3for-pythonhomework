\documentclass[11pt,a4paper]{article}
\usepackage[UTF8]{ctex}

\usepackage[margin=2cm, top=2.2cm, bottom=2cm]{geometry}
\usepackage{setspace}
\setstretch{1.05}
\setlength{\parskip}{0.2em}
\setlength{\headheight}{14.5pt}
\usepackage{listings}
\usepackage{xcolor}

\definecolor{codegreen}{rgb}{0,0.6,0}
\definecolor{codegray}{rgb}{0.5,0.5,0.5}
\definecolor{codepurple}{rgb}{0.58,0,0.82}
\definecolor{backcolour}{rgb}{0.95,0.95,0.95}
\definecolor{myred}{rgb}{0.8,0,0}

\lstdefinestyle{mystyle}{
    backgroundcolor=\color{backcolour},
    commentstyle=\color{codegreen},
    keywordstyle=\color{blue},
    numberstyle=\tiny\color{codegray},
    stringstyle=\color{codepurple},
    basicstyle=\ttfamily\scriptsize,
    breaklines=true,
    keepspaces=true,
    numbers=left,
    numbersep=3pt,
    tabsize=2,
    language=Python,
    frame=single,
    xleftmargin=1em,
    framexleftmargin=0.5em
}
\lstset{style=mystyle}

% ==================== 图表支持 ====================
\usepackage{graphicx}
\usepackage{tikz}
\usetikzlibrary{shapes.geometric, arrows, positioning, fit, calc, shapes.arrows}

% ==================== 超链接 ====================
\usepackage{hyperref}
\hypersetup{colorlinks=true, linkcolor=blue, urlcolor=cyan}

% ==================== 其他 ====================
\usepackage{enumitem}
\usepackage{booktabs}
\usepackage{fancyhdr}
\usepackage{float}
\usepackage{multicol}
\usepackage{tabularx}

% 紧凑列表
\setlist{nosep, leftmargin=1.5em}

% 页眉页脚
\pagestyle{fancy}
\fancyhf{}
\rhead{\small Project 2: Alien Invasion}
\lhead{\small Python Programming}
\rfoot{\small Page \thepage}

% ============================================
\begin{document}

% ==================== 紧凑标题头 ====================
\begin{center}
{\LARGE\bfseries Python程序设计 \quad 实验报告}\\[0.3cm]
{\large\bfseries Project 2: 外星人入侵游戏}\\[0.3cm]
\begin{tabular}{cccc}
\textbf{姓名:}陈铄涵 & \textbf{学号:}2024140014 & \textbf{课程:}Python程序设计 & \textbf{日期:}\today
\end{tabular}\\[0.1cm]
\textbf{GitHub:} \url{https://github.com/Csh0601/-}
\end{center}
\vspace{-0.3cm}
\hrule
\vspace{0.2cm}

% ==================== 正文开始 ====================
\section{项目概述与工作内容}

本项目基于《Python Crash Course》的外星人入侵游戏,\textbf{\textcolor{myred}{独立设计并实现了4大增强功能}},提升游戏可玩性。

\vspace{-0.3cm}
\begin{figure}[H]
\centering
\begin{tikzpicture}[
    node distance=0.6cm,
    startstop/.style={rectangle, rounded corners, minimum width=1.8cm, minimum height=0.6cm, text centered, draw=black, fill=red!20, font=\scriptsize},
    process/.style={rectangle, minimum width=1.8cm, minimum height=0.6cm, text centered, draw=black, fill=blue!15, font=\scriptsize},
    myprocess/.style={rectangle, minimum width=1.8cm, minimum height=0.6cm, text centered, draw=red!70, fill=red!10, line width=1pt, font=\scriptsize\bfseries},
    decision/.style={diamond, minimum width=1cm, minimum height=0.5cm, text centered, draw=black, fill=green!15, font=\scriptsize, aspect=2},
    arrow/.style={thick,->,>=stealth}
]
    \node (start) [startstop] {开始};
    \node (init) [process, right=of start] {初始化};
    \node (event) [process, right=of init] {事件检测};
    \node (update) [myprocess, right=of event] {更新状态};
    \node (render) [process, right=of update] {渲染画面};
    \node (check) [decision, right=of render] {退出?};
    \node (end) [startstop, right=of check] {结束};
    
    \draw [arrow] (start) -- (init);
    \draw [arrow] (init) -- (event);
    \draw [arrow] (event) -- (update);
    \draw [arrow] (update) -- (render);
    \draw [arrow] (render) -- (check);
    \draw [arrow] (check) -- node[above, font=\tiny] {是} (end);
    \draw [arrow] (check.south) -- ++(0,-0.5) -| node[near start, right, font=\tiny] {否} (event.south);
\end{tikzpicture}
\caption{\textcolor{myred}{游戏主循环流程图}(红框为本人重点修改:状态更新逻辑)}
\end{figure}

\vspace{-0.5cm}
\subsection*{\textcolor{myred}{本人完成的主要工作(红色标注)}}
\vspace{-0.2cm}

\begin{table}[H]
\centering\small
\begin{tabularx}{\textwidth}{|l|X|l|}
\hline
\textbf{\textcolor{myred}{功能模块}} & \textbf{实现内容} & \textbf{涉及文件} \\
\hline
\textcolor{myred}{难度选择系统} & 三档难度(Easy/Normal/Hard),动态调整生命、子弹、速度、道具掉率 & settings.py, button.py \\
\hline
\textcolor{myred}{多类型外星人} & 普通(绿)、快速(黄)、强化(红)三种类型,不同速度和分值 & alien.py \\
\hline
\textcolor{myred}{道具掉落系统} & 护盾(无敌)、火力(增强)、速度(加速)、生命(+1)四种道具 & powerup.py (新建) \\
\hline
\textcolor{myred}{Boss战系统} & 每3关出现Boss,血条显示,需多次击中,击败得500分 & boss.py (新建) \\
\hline
\textcolor{myred}{四方向移动} & 飞船支持上下左右全方向移动 & ship.py \\
\hline
\textcolor{myred}{状态管理} & 菜单→难度选择→游戏→暂停→结束的完整状态机 & alien\_invasion.py \\
\hline
\end{tabularx}
\end{table}

\vspace{-0.3cm}
\begin{figure}[H]
\centering
\begin{tikzpicture}[
    node distance=1.2cm and 1.5cm,
    state/.style={rectangle, rounded corners, minimum width=2cm, minimum height=0.7cm, text centered, draw=black, fill=blue!20, font=\small},
    mystate/.style={rectangle, rounded corners, minimum width=2cm, minimum height=0.7cm, text centered, draw=red!70, fill=red!15, line width=1.2pt, font=\small\bfseries},
    arrow/.style={thick,->,>=stealth}
]
    \node (menu) [state] {主菜单};
    \node (diff) [mystate, right=of menu] {难度选择};
    \node (play) [mystate, right=of diff] {游戏进行};
    \node (pause) [mystate, above=0.8cm of play] {暂停};
    \node (over) [state, right=of play] {游戏结束};
    
    \draw [arrow] (menu) -- node[above,font=\tiny] {Play} (diff);
    \draw [arrow] (diff) -- node[above,font=\tiny] {选择} (play);
    \draw [arrow] (play) -- (pause);
    \draw [arrow] (pause) -- node[right,font=\tiny] {P键} (play);
    \draw [arrow] (play) -- node[above,font=\tiny] {生命=0} (over);
    \draw [arrow] (over.south) -- ++(0,-0.4) -| node[near start, below, font=\tiny] {R键} (diff.south);
    \draw [arrow] (over.north) -- ++(0,0.6) -| node[near start, above, font=\tiny] {ESC} (menu.north);
\end{tikzpicture}
\caption{\textcolor{myred}{游戏状态机流程图}(红框为本人新增:难度选择、暂停功能)}
\end{figure}

% ==================== 系统架构 ====================
\section{系统架构设计}

\vspace{-0.2cm}
\begin{figure}[H]
\centering
\begin{tikzpicture}[
    class/.style={rectangle, draw=black, fill=blue!10, text width=1.8cm, minimum height=0.7cm, align=center, font=\tiny\ttfamily},
    myclass/.style={rectangle, draw=red!70, fill=red!10, text width=1.8cm, minimum height=0.7cm, align=center, font=\tiny\ttfamily, line width=1.2pt},
    newclass/.style={rectangle, draw=red!70, fill=yellow!20, text width=1.8cm, minimum height=0.7cm, align=center, font=\tiny\ttfamily, line width=1.2pt},
    arrow/.style={->, thick, >=stealth}
]
    \node[class] (ai) at (0,0) {AlienInvasion\\主控制器};
    \node[myclass] (ship) at (-4.5,-1.3) {Ship\\飞船};
    \node[class] (bullet) at (-2.5,-1.3) {Bullet\\子弹};
    \node[myclass] (alien) at (-0.5,-1.3) {Alien\\外星人};
    \node[newclass] (boss) at (1.5,-1.3) {Boss\\(新建)};
    \node[newclass] (powerup) at (3.5,-1.3) {PowerUp\\(新建)};
    \node[myclass] (settings) at (-3.5,-2.6) {Settings\\设置};
    \node[class] (stats) at (-1.5,-2.6) {GameStats\\统计};
    \node[class] (sb) at (0.5,-2.6) {Scoreboard\\记分板};
    \node[myclass] (btn) at (2.5,-2.6) {Button\\按钮};
    
    \foreach \target in {ship, bullet, alien, boss, powerup, settings, stats, sb, btn}
        \draw[arrow] (ai) -- (\target);
    
    \node[font=\tiny] at (4.2,-2.6) {\textcolor{red}{红框}=修改};
    \node[font=\tiny] at (4.2,-3) {\textcolor{yellow!70!orange}{黄框}=新建};
\end{tikzpicture}
\caption{游戏类结构图(10个核心类,\textcolor{myred}{本人修改/新建6个})}
\end{figure}

\vspace{-0.4cm}
\subsection{项目文件结构}
\vspace{-0.2cm}

\begin{table}[H]
\centering\scriptsize
\begin{tabular}{|l|l|c|}
\hline
\textbf{文件} & \textbf{功能描述} & \textbf{状态} \\
\hline
alien\_invasion.py & 主程序,游戏循环与状态机 & \textcolor{myred}{修改} \\
\hline
\textcolor{myred}{settings.py} & \textcolor{myred}{游戏设置,含难度配置系统} & \textcolor{myred}{修改} \\
\hline
\textcolor{myred}{ship.py} & \textcolor{myred}{飞船类,支持四方向移动+速度加成} & \textcolor{myred}{修改} \\
\hline
\textcolor{myred}{alien.py} & \textcolor{myred}{外星人类,支持3种类型} & \textcolor{myred}{修改} \\
\hline
\textcolor{myred}{powerup.py} & \textcolor{myred}{道具系统(护盾/火力/速度/生命)} & \textcolor{myred}{\textbf{新建}} \\
\hline
\textcolor{myred}{boss.py} & \textcolor{myred}{Boss类(血条、特殊移动模式)} & \textcolor{myred}{\textbf{新建}} \\
\hline
\textcolor{myred}{button.py} & \textcolor{myred}{按钮类,支持难度选择界面} & \textcolor{myred}{修改} \\
\hline
bullet.py, game\_stats.py, scoreboard.py & 子弹类、游戏统计、记分板 & 原有 \\
\hline
\end{tabular}
\end{table}

\vspace{-0.3cm}
\subsection{道具系统工作流程}
\vspace{-0.2cm}

\begin{figure}[H]
\centering
\begin{tikzpicture}[
    node distance=0.5cm and 0.8cm,
    process/.style={rectangle, rounded corners, minimum width=1.4cm, minimum height=0.5cm, text centered, draw=black, fill=blue!15, font=\tiny},
    myprocess/.style={rectangle, rounded corners, minimum width=1.4cm, minimum height=0.5cm, text centered, draw=red!70, fill=red!10, line width=1pt, font=\tiny},
    decision/.style={diamond, minimum width=0.8cm, minimum height=0.4cm, text centered, draw=black, fill=green!15, font=\tiny, aspect=1.8},
    arrow/.style={thick,->,>=stealth,font=\tiny}
]
    \node (kill) [process] {击杀外星人};
    \node (rand) [decision, right=of kill] {随机};
    \node (drop) [myprocess, right=of rand] {生成道具};
    \node (fall) [process, right=of drop] {道具下落};
    \node (col) [decision, right=of fall] {碰撞?};
    \node (act) [myprocess, right=of col] {激活效果};
    \node (timer) [myprocess, right=of act] {计时结束};
    
    \draw [arrow] (kill) -- (rand);
    \draw [arrow] (rand) -- node[above] {15\%} (drop);
    \draw [arrow] (rand.south) -- ++(0,-0.3) -- node[below] {85\%} ++(1.5,0) |- (kill.south);
    \draw [arrow] (drop) -- (fall);
    \draw [arrow] (fall) -- (col);
    \draw [arrow] (col) -- node[above] {是} (act);
    \draw [arrow] (col.south) -- ++(0,-0.3) -- node[below] {否} ++(-1.2,0) |- (fall.south);
    \draw [arrow] (act) -- (timer);
\end{tikzpicture}
\caption{\textcolor{myred}{道具系统流程图}(本人独立设计实现)}
\end{figure}

% ==================== 核心代码 ====================
\vspace{-0.4cm}
\section{核心代码与技术亮点}

\vspace{-0.2cm}
\begin{multicols}{2}
\subsection*{\textcolor{myred}{1. 难度动态配置}}
\begin{lstlisting}
def set_difficulty(self, diff):
  if diff == 'easy':
    self.ship_limit = 5
    self.base_alien_speed = 0.7
  elif diff == 'hard':
    self.ship_limit = 2
    self.base_alien_speed = 1.5
\end{lstlisting}

\subsection*{\textcolor{myred}{2. 外星人类型工厂}}
\begin{lstlisting}
ALIEN_TYPES = {
  'normal': {'color':(100,200,100),
             'speed':1.0, 'pts':50},
  'fast':   {'color':(255,200,50),
             'speed':1.8, 'pts':75},
  'strong': {'color':(220,80,80),
             'speed':0.7, 'pts':100}
}
\end{lstlisting}
\end{multicols}

\vspace{-0.5cm}
\begin{multicols}{2}
\subsection*{\textcolor{myred}{3. 道具效果激活}}
\begin{lstlisting}
def _activate_effect(self, type):
  if type == 'shield':
    self.effects['shield'] = 300
  elif type == 'life':
    self.stats.ships_left += 1
  elif type == 'speed':
    self.settings.speed_boost=1.5
\end{lstlisting}

\subsection*{\textcolor{myred}{4. Boss血条与伤害}}
\begin{lstlisting}
def take_damage(self, dmg=1):
  self.health -= dmg
  self.is_hurt = True
  if self.health <= 0:
    self.alive = False
    return True  # Boss defeated
  return False
\end{lstlisting}
\end{multicols}

% ==================== 游戏截图 ====================
\vspace{-0.3cm}
\section{游戏运行效果展示}

\vspace{-0.2cm}
\begin{figure}[H]
\centering
\begin{minipage}{0.48\textwidth}
\centering
\includegraphics[width=\textwidth]{正常运行.png}
\caption*{(a) 道具效果:左上角BULLET增强倒计时}
\end{minipage}
\hfill
\begin{minipage}{0.48\textwidth}
\centering
\includegraphics[width=\textwidth]{多样的怪物.png}
\caption*{(b) 战斗场景:外星人被击杀后的动态布局}
\end{minipage}
\caption{游戏运行实际效果(左:道具系统生效;右:激烈战斗中)}
\end{figure}

% ==================== 可玩性亮点 ====================
\vspace{-0.3cm}
\section{可玩性亮点与技术特点}

\noindent\textbf{\textcolor{myred}{可玩性亮点:}}难度递进(每关1.1x加速) | Boss挑战(每3关出现) | 道具策略(护盾/火力/速度/生命) | 三档难度选择(Easy/Normal/Hard)

\vspace{0.1cm}
\noindent\textbf{技术特点:}面向对象(10类) | 精灵组管理(pygame.sprite.Group) | 状态机(5状态) | 帧计时器 | 工厂模式

% ==================== 运行说明 ====================
\vspace{-0.3cm}
\section{运行说明}

\begin{lstlisting}[language=bash]
pip install pygame          # 安装依赖
python alien_invasion.py    # 启动游戏
\end{lstlisting}

\vspace{-0.3cm}
\noindent\textbf{操作:}方向键移动 | 空格发射 | P暂停 | R重启 | Q/ESC退出(需英文输入法)

% ==================== 项目总结 ====================
\vspace{-0.2cm}
\section{项目总结}

通过本项目,深入实践了:\textbf{面向对象设计}(类的封装与继承)、\textbf{游戏状态机}(多状态切换管理)、\textbf{精灵系统}(pygame.sprite碰撞检测)、\textbf{配置驱动}(难度参数化设计)。

% ==================== 教师评语 ====================
\vspace{0.5cm}
\section{教师评语}

\noindent\fbox{\parbox{\dimexpr\textwidth-2\fboxsep-2\fboxrule\relax}{
\vspace{3cm}
\hfill \textbf{评分:}\underline{\hspace{2cm}} \quad \textbf{日期:}\underline{\hspace{3cm}}
\vspace{0.3cm}
}}

\end{document}
